\documentclass[12pt]{article}
\usepackage[utf8]{inputenc}
\usepackage[russian]{babel}
\usepackage{amsmath}
\usepackage{amssymb}
\usepackage{geometry}
\usepackage{mathtools}

\geometry{a4paper, left=2cm, right=2cm, top=2cm, bottom=2cm}

\title{Основные формулы для системы Bluetooth навигации в помещениях}
\author{[Имя автора]}
\date{\today}

\begin{document}

\maketitle

\section{Модель распространения сигнала}

Для расчета расстояния по RSSI используется модифицированная модель:

\begin{equation}
RSSI = A - 10n \log_{10}(d)
\end{equation}

Где:
\begin{itemize}
    \item $RSSI$ - уровень принимаемого сигнала
    \item $A$ - уровень сигнала на расстоянии 1 метра (TX Power)
    \item $n$ - коэффициент затухания (обычно 2-4 в помещениях)
    \item $d$ - расстояние до маячка в метрах
\end{itemize}

Отсюда можно выразить расстояние:

\begin{equation}
d = 10^{\frac{RSSI - A}{-10n}}
\end{equation}

\section{Трилатерация}

Для трех маячков с координатами $(x_1, y_1)$, $(x_2, y_2)$, $(x_3, y_3)$ и расстояниями до них $d_1$, $d_2$, $d_3$, решаем систему уравнений:

\begin{equation}
\begin{cases}
(x - x_1)^2 + (y - y_1)^2 = d_1^2 \\
(x - x_2)^2 + (y - y_2)^2 = d_2^2 \\
(x - x_3)^2 + (y - y_3)^2 = d_3^2
\end{cases}
\end{equation}

Преобразуем в систему линейных уравнений:

\begin{equation}
\begin{cases}
2(x_2 - x_1)x + 2(y_2 - y_1)y = (x_2^2 - x_1^2) + (y_2^2 - y_1^2) - (d_2^2 - d_1^2) \\
2(x_3 - x_2)x + 2(y_3 - y_2)y = (x_3^2 - x_2^2) + (y_3^2 - y_2^2) - (d_3^2 - d_2^2)
\end{cases}
\end{equation}

Решение системы:

\begin{align}
x &= \frac{(y_3 - y_1)(d_2^2 - d_1^2 - x_2^2 + x_1^2 - y_2^2 + y_1^2) - (y_2 - y_1)(d_3^2 - d_1^2 - x_3^2 + x_1^2 - y_3^2 + y_1^2)}{2[(x_2 - x_1)(y_3 - y_1) - (x_3 - x_1)(y_2 - y_1)]} \\
y &= \frac{(x_3 - x_1)(d_2^2 - d_1^2 - x_2^2 + x_1^2 - y_2^2 + y_1^2) - (x_2 - x_1)(d_3^2 - d_1^2 - x_3^2 + x_1^2 - y_3^2 + y_1^2)}{2[(y_2 - y_1)(x_3 - x_1) - (y_3 - y_1)(x_2 - x_1)]}
\end{align}

\section{Многомерная трилатерация (метод наименьших квадратов)}

Когда доступно более 3 маячков, используется метод наименьших квадратов для улучшения точности:

\begin{equation}
\min_{x,y} \sum_{i=1}^{n} w_i \cdot (r_i - \sqrt{(x - x_i)^2 + (y - y_i)^2})^2
\end{equation}

Где:
\begin{itemize}
    \item $w_i$ - вес i-го маячка (обратно пропорционален ошибке измерения)
    \item $r_i$ - измеренное расстояние до i-го маячка
    \item $(x_i, y_i)$ - координаты i-го маячка
\end{itemize}

\section{Фильтрация сигналов}

\subsection{Скользящее среднее}

Для сглаживания RSSI-сигнала:

\begin{equation}
\overline{RSSI}_k = \frac{1}{N} \sum_{i=k-N+1}^{k} RSSI_i
\end{equation}

\subsection{Фильтр Калмана}

Для фильтрации координат пользователя:

Система уравнений:
\begin{align}
\hat{x}_k^- &= F_k \hat{x}_{k-1} + B_k u_k \\
P_k^- &= F_k P_{k-1} F_k^T + Q_k
\end{align}

Коэффициент Калмана:
\begin{equation}
K_k = \frac{P_k^-}{P_k^- + R_k}
\end{equation}

Обновление оценки:
\begin{align}
\hat{x}_k &= \hat{x}_k^- + K_k(z_k - H_k \hat{x}_k^-) \\
P_k &= (I - K_k H_k) P_k^-
\end{align}

Где:
\begin{itemize}
    \item $\hat{x}_k$ - оценка состояния в момент k
    \item $P_k$ - ковариационная матрица ошибки
    \item $Q_k$ - ковариация процесса шума
    \item $R_k$ - ковариация измерительного шума
\end{itemize}

\section{Алгоритм A*}

Функция оценки стоимости:
\begin{equation}
f(n) = g(n) + h(n)
\end{equation}

Где:
\begin{itemize}
    \item $f(n)$ - оценка полной стоимости пути через узел n
    \item $g(n)$ - стоимость пути от начального узла до узла n
    \item $h(n)$ - эвристическая оценка стоимости от узла n до цели
\end{itemize}

\subsection{Эвристическая функция (Евклидово расстояние)}

\begin{equation}
h(n) = \sqrt{(x_n - x_{goal})^2 + (y_n - y_{goal})^2}
\end{equation}

\section{Модель точности определения местоположения}

\subsection{Оценка точности}

Точность определения местоположения зависит от количества доступных маячков, геометрической конфигурации маячков и качества измерений RSSI:

\begin{equation}
\sigma_{pos} = \sqrt{\frac{\sum_{i=1}^{n} \sigma_i^2}{n}}
\end{equation}

Где $\sigma_i$ - ошибка определения расстояния до i-го маячка.

\subsection{Геометрический фактор (GDOP)}

\begin{equation}
GDOP = \sqrt{\text{trace}((H^T R^{-1} H)^{-1})}
\end{equation}

Где:
\begin{itemize}
    \item $H$ - матрица геометрии (jacobian матрица)
    \item $R$ - ковариационная матрица измерений
\end{itemize}

\end{document}